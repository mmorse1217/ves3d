\documentclass[11pt]{article}

\usepackage[fleqn,reqno]{amsmath}
\usepackage{amsfonts, amssymb, stmaryrd, xifthen}

\def\defeq{\mathrel{\mathop :}=}
\def\R{\mathbb{R}}

\usepackage{algorithmic,algorithm}
%% \newlength\commLen
%% \newlength\textLen
%% \newlength\resLen
%% \renewcommand{\algorithmiccomment}[1]{
%%         \settowidth\textLen{\footnotesize\tt// #1}
%%         \setlength\resLen{\the\commLen}
%%         \addtolength\resLen{-\the\textLen}
%%         \hfill{\footnotesize\tt// #1}\hspace{\the\resLen}}

\begin{document}

In our red blood cell simulation, at any time step there is a
computation stage involving a series of matrix-matrix multiplications.
Let $p$ denote the independent size variable, $N$ the number of RBC's,
and let $A^i (i=0, \dots, p), B$, and $C$ denote the matrices
involved.  Matrices $A^i$ are fixed, while the matrix $B$ is the
position of RBC's that changes between time steps. Letting $M \defeq
2p(p+1)$, then we have $A^i \in \R^{M\times M}$, $B, C \in \R^{M\times
 N}$. The pseudocode for that portion of the code would be

\begin{algorithm}[!hbt]
  \begin{algorithmic}
    \FOR{$i=0$ to $p$}
      \FOR{$j=0$ to $2p-1$}

      \STATE $A^{i.j} \leftarrow$ permute $A^i$
      \COMMENT{The permutation formula is given below}

      \STATE $C\leftarrow A^{i,j} B$

      \STATE {\footnotesize \tt{//Perform some processing on C}}

      \ENDFOR
    \ENDFOR
  \end{algorithmic}
\end{algorithm}

The permutation step mentioned in the pseudocode involves exchanging
columns of $A^i$. For any fixed $j\;(j=0,\dots,2p-1)$ it is defined as
\[A^{i,j}(n,m) = A^i\left(n, \left\lfloor \frac{m}{2p} \right\rfloor +
  m\mod 2p - j \right),\] where we used Matlab's notation for
indexing. It is understood that $m\mod 2p - j$ is to be positive and
between $0$ and $2p-1$. When negative it is shifted by $2p$.

\end{document}
